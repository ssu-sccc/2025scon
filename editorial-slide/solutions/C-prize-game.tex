\def\probno{C}
\def\probtitle{특별상 눈치게임}

\section{\probno{}. \probtitle{}}

\begin{frame} % No title at first slide
    \sectiontitle{\probno{}}{\probtitle{}}
    \sectionmeta{
        \texttt{bruteforcing}\\
        출제진 의도 -- \textbf{\color{acbronze}Easy}
    }
    \begin{itemize}
        \item 처음 푼 팀: \textbf{TeamRAM}, 26분
        \item 문제 아이디어: 최도현
        \item 문제 작업: 박근형
    \end{itemize}
\end{frame}

\begin{frame}{\probno{}. \probtitle{}}
    \begin{itemize}
        \item 정수 $i$를 고른 팀의 개수를 $cnt[i]$라고 합시다.
        \item 어떤 조합 $(x, y, z)$를 고르는 것은 $cnt[x]$와 $cnt[y]$와 $cnt[z]$에 각각 $1$씩 더한 것과 같습니다.
        \vspace{3mm}
        \item $cnt$ 배열을 뒤에서부터 보면서, 처음으로 $cnt[i] = 1$인 $i$를 고른 팀이 특별상을 받게 됩니다.
        \item 이 과정에서 필요한 연산의 횟수는 최대 100번입니다.
        \item 정수 3개를 고르는 방법의 수는 $C = \frac{100\times 99 \times 98}{6} \approx 1.6 \times 10^5$입니다.
        \vspace{3mm}
        \item 따라서 가능한 $C$가지 방법을 모두 순회하면서, 특별상을 받을 수 있는지 확인하면 됩니다.
        \item 총 연산 횟수는 $100C \approx 1.6 \times 10^7$ 정도이므로, 제한 시간 안에 정답을 구할 수 있습니다.
    \end{itemize}
\end{frame}
