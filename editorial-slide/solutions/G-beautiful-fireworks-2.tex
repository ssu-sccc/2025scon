\def\probno{G}
\def\probtitle{불꽃놀이의 아름다움 2}

\section{\probno{}. \probtitle{}}

\begin{frame} % No title at first slide
    \sectiontitle{\probno{}}{\probtitle{}}
    \sectionmeta{
        \texttt{bipartite\_graph, dfs}\\
        출제진 의도 -- \textbf{\color{acgold}Hard}
    }
    \begin{itemize}
        \item 처음 푼 팀: \textbf{와장창특공대}, 57분
        \item 문제 아이디어: 진민성
        \item 문제 작업: 박근형
    \end{itemize}
\end{frame}

\begin{frame}{\probno{}. \probtitle{}}
    \begin{itemize}
        \item 문제에서 주어진 구조를 정점과 간선이 $N$개인 연결 그래프로 생각합시다.
        \item 이러한 그래프는 트리에 간선 1개를 추가해서 만들 수 있습니다.
        \item 따라서 이 그래프는 \textbf{단순 사이클} 하나와, 그 주변에 몇 개의 \textbf{트리}가 붙어 있는 형태입니다.
        \vspace{3mm}
        \item \textbf{단순 사이클}은 길이가 짝수일 경우 2개, 홀수일 경우 3개의 색으로 칠할 수 있습니다.
        \item \textbf{트리}는 깊이를 2로 나눈 나머지를 이용하면 항상 2개의 색으로 칠할 수 있습니다.
        \item 따라서 문제의 정답은 사이클의 길이를 2로 나눈 나머지를 이용해 결정할 수 있습니다.
        \item DFS 등을 이용해 사이클의 길이를 구하면 $O(N)$ 시간에 문제를 해결할 수 있습니다.
    \end{itemize}
\end{frame}

\begin{frame}{\probno{}. \probtitle{}}
    \begin{itemize}
        \item 여담으로, 어떤 그래프를 2개의 색으로 칠할 수 있는 것과 그래프에 홀수 길이 사이클이 없는 것은 동치입니다.
        \item 따라서 정답이 항상 2 또는 3이라는 것을 관찰했다면, 그래프가 이분 그래프인지 판별하는 것으로 더 편하게 구현할 수도 있습니다.
    \end{itemize}
\end{frame}
