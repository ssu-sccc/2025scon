\def\probno{E}
\def\probtitle{even하게 익은 SCON}

\section{\probno{}. \probtitle{}}

\begin{frame} % No title at first slide
    \sectiontitle{\probno{}}{\probtitle{}}
    \sectionmeta{
        \texttt{dp, mathmatics}\\
        출제진 의도 -- \textbf{\color{acsilver}Medium}
    }
    \begin{itemize}
        \item 처음 푼 팀: \textbf{EME}, 21분
        \item 문제 아이디어: 조문성
        \item 문제 작업: 조문성
    \end{itemize}
\end{frame}

\begin{frame}{\probno{}. \probtitle{} - 동적 계획법}
    \begin{itemize}
        \item 문자를 한 글자씩 추가하는 방식으로 문자열을 구성한다고 생각해 봅시다.
        \item $S$와 $C$의 실제 개수는 중요하지 않고, 홀짝 여부만 중요하다는 것을 알 수 있습니다.
        \item 그러므로 다음과 같은 점화식을 생각해 볼 수 있습니다.
        \begin{itemize}
            \item $D(i, 0) :=$ 길이가 $i$이고, $S$와 $C$의 개수가 짝수인 문자열의 수
            \item $D(i, 1) :=$ 길이가 $i$이고, $S$와 $C$의 개수가 홀수인 문자열의 수
        \end{itemize}
    \end{itemize}
\end{frame}

\begin{frame}{\probno{}. \probtitle{} - 동적 계획법}
    \begin{itemize}
        \item 점화식은 문자열의 맨 뒤에 추가하는 문자에 따라 두 가지 경우로 나눠서 계산할 수 있습니다.
        \begin{itemize}
            \item $S$ 또는 $C$를 추가하는 경우: $D(i, j) \leftarrow 2 \times D(i-1, 1-j)$
            \item $S$와 $C$가 아닌 다른 문자를 추가하는 경우: $D(i, j) \leftarrow 24 \times D(i-1, j)$
        \end{itemize}
        \item 따라서 아래 식을 이용해 답을 $O(N)$ 시간에 문제의 정답인 $D(N, 0)$을 구할 수 있습니다.
        \begin{itemize}
            \item $D(i, 0) = 24 \times D(i-1, 0) + 2 \times D(i-1, 1)$
            \item $D(i, 1) = 24 \times D(i-1, 1) + 2 \times D(i-1, 0)$
        \end{itemize}
    \end{itemize}
\end{frame}

\begin{frame}{\probno{}. \probtitle{} - 조합론}
    \begin{itemize}
        \item 별해로 이항 계수와 모듈러 곱셈 역원으로 정답을 구할 수 있습니다.
        \item $S, C$의 개수를 $2i$개로 고정시켜 봅시다.
        \item 이때 가능한 문자열의 개수는 $\binom{n}{2i} \cdot 2^{2i} \cdot 24^{n - 2i}$ 개 입니다.
        \item 따라서 정답은 $\sum_{i=0}^{\lfloor n/2 \rfloor} \binom{n}{2i} \cdot 2^{2i} \cdot 24^{n - 2i}$ 입니다.
        \vspace{3mm}
        \item $2^{2i}$와 $24^{n-2i}$는 $O(n)$ 시간에 모두 계산할 수 있습니다.
        \item 모듈러 곱셈 역원을 이용하면 $O(n + \log P)$ 시간의 전처리를 통해 이항 계수를 상수 시간에 구할 수 있습니다.
        \item 따라서 전체 시간 복잡도는 $O(n + \log P)$입니다. (단, $P = 10^9+7$)
    \end{itemize}
\end{frame}

\begin{frame}{\probno{}. \probtitle{} - 일반항}
    \begin{itemize}
        \item $A, B, \cdots, Z$를 변수로 생각합시다.
        \item $(A+B+\cdots+Z)^n$ 을 전개한 각 항은 길이 $n$인 서로 다른 문자열입니다.
        \begin{itemize}
        {\small
            \item 예) $(A+B+\cdots+Z)^2 = AA+AB+ \cdots+ AZ+ BA+ \cdots + ZA+ZB+ \cdots+ZZ$
        }
        \end{itemize}
        \item $S = C = -1$, 나머지 변수에 $1$을 대입하면 각 문자열의 값은 ${(-1)}^{(S,\ C\textrm{가 등장한 횟수})}$입니다.
        \begin{itemize}
        {\small
            \item 예) $SCON= (-1)\times(-1)\times1\times1 = (-1)^2, ASDF = 1 \times (-1) \times 1 \times 1 = (-1)^1$
        }
            \item 문자열의 값은 $S,\ C$가 홀수번 등장하면 값은 $-1$, 짝수번 등장하면 $1$이 됩니다.
            \item \phantom{.}
        \end{itemize}
        \vspace{-10.5mm}
        \begin{align*}
            22^n & = (A+B+\cdots+Z)^n = \textrm{길이 }n\textrm{인 서로 다른 모든 문자열 값의 합} \\
            & = (S,\ C\textrm{가 짝수번 등장하는 문자열 수}) - (S,\ C\textrm{가 홀수번 등장하는 문자열 수}) \\
        26^n &= (S,\ C\textrm{가 짝수번 등장하는 문자열 수}) + (S,\ C\textrm{가 홀수번 등장하는 문자열 수})
        \end{align*}
        \vspace{-8mm}
        \item 두 식을 연립하면 $(S,\ C\textrm{가 짝수번 등장하는 문자열 수}) = \frac{26^n+22^n}{2}$ 입니다.
    \end{itemize}
\end{frame}
