\def\probtitle{even하게 익은 SCON}
\def\probno{E} % 문제 번호

\begin{problem}{\probno{}. \probtitle{}}

SCON과 SCCC를 바라보던 문성이는 흥미로운 규칙을 발견했다! 바로 두 문자열에서 문자 `S'와 `C'의 개수를 합하면 항상 짝수가 된다는 것이다. 이를 본 문성이는 문득 알파벳 대문자로 구성된 길이가 $N$인 문자열 중에서 `S'와 `C'의 개수의 합이 짝수인 문자열이 몇 개나 될지 궁금해졌다.

문성이의 궁금증을 해결해 주자!

\InputFile

첫째 줄에 문자열의 길이를 의미하는 정수 $N$이 주어진다.

\OutputFile

알파벳 대문자로 구성된 길이가 $N$인 문자열 중 `S', `C'의 개수의 합이 짝수인 문자열의 개수를 $1\,000\,000\,007(=10^9+7)$로 나눈 나머지를 출력하라.

\Constraints

\begin{itemize}[topsep=0pt,noitemsep]
    \item $1 \le N \le 1\,000\,000$
    \item 입력으로 주어지는 수는 모두 정수이다.
\end{itemize}

\Example

\begin{example}
    \exmpfile{./example/01.in.txt}{./example/01.out.txt}%
    \exmpfile{./example/02.in.txt}{./example/02.out.txt}%
    \exmpfile{./example/03.in.txt}{./example/03.out.txt}%
\end{example}

\end{problem}