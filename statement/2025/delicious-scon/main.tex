\def\probtitle{맛있는 스콘 만들기}
\def\probno{J} % 문제 번호

\begin{problem}{\probno{}. \probtitle{}}

도현이는 SCON 참가자들에게 나눠줄 간식으로 스콘을 만들기로 했다.  

매 시각마다 스콘이 가장 맛있게 구워지는 최적의 온도가 존재한다는 사실을 알고 있는 도현이는 시각 $0$부터 시작하여 시각 $N-1$까지 오븐의 온도를 총 $N$번 조절하여 스콘을 굽기로 했다.

스콘을 만드는 데 사용할 오븐은 $1$부터 $M$까지의 정수 온도로 조절이 가능하다.  
각 시각마다 한 번만 오븐의 온도를 조절할 수 있는데, 시각 $0$에서는 자유롭게 오븐의 온도를 조절할 수 있지만, 다른 시각에서는 기존 온도에서 $C$의 정수 배만큼만 온도를 높이거나 낮출 수 있다. 또, 한 번에 $D$를 초과해서 온도를 높이거나 낮출 수 없다. 예를 들어, 시각 $t$에서 오븐의 온도를 $182$로 조절했고, $C=5, D=10$일 경우, 시각 $t+1$에서 조절 가능한 온도는 $172, 177, 182, 187, 192$뿐이다.  

시각 $t$에서의 최적의 온도를 $b_t$, 시각 $t$에서 조절한 오븐의 온도를 $k_t$라고 하면 시각 $t+1$에서 스콘의 맛은 $M - |b_t - k_t|$만큼 증가한다. 

시각 $0$부터 시각 $N-1$까지의 최적의 온도가 주어졌을 때, 시각 $N$에 완성되는 스콘의 맛의 최댓값을 구해보자. 처음 스콘을 오븐에 넣었을 때의 시각은 $0$이며, 스콘의 맛은 $0$이다. 모든 시각은 정수 시각만 고려한다.

\InputFile
첫째 줄에 네 정수 $N, M, C, D$가 공백으로 구분되어 주어진다.

둘째 줄에 각 시각의 최적의 온도를 의미하는 $N$개의 정수 $b_0, b_1, ... , b_{N-1}$이 주어진다.

\OutputFile

시각 $N$에 완성되는 스콘의 맛의 최댓값을 출력한다.

\Constraints

\begin{itemize}[topsep=0pt,noitemsep]
    \item $1 \leq N \leq 200$
    \item $1 \leq M \leq 25\,000$
    \item $1 \leq C \leq D \leq M$
    \item $D$는 $C$의 배수이다.
    \item $1 \le b_i \le M$ ($0 \le i \le N-1$)
    \item 입력으로 주어지는 수는 모두 정수이다.
\end{itemize}

\Examples

\begin{example}
\exmpfile{./example/01.in.txt}{./example/01.out.txt}%
\exmpfile{./example/02.in.txt}{./example/02.out.txt}%
\end{example}

\Note
첫 번째 예시에서 오븐의 온도를 차례대로 3, 7, 3으로 조절했을 경우, 스콘의 맛은 8 + 8 + 6 = 22가 되며, 이보다 더 맛있게 스콘을 만드는 방법은 존재하지 않음을 증명할 수 있다.

\end{problem}