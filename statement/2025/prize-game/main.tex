\def\probtitle{특별상 눈치게임}
\def\probno{C} % 문제 번호

\begin{problem}{\probno{}. \probtitle{}}

당신의 팀을 포함해 총 $N+1$개의 팀이 특별상을 받기 위해 눈치게임을 시작했다! 눈치게임의 규칙은 다음과 같다.

\begin{enumerate}[topsep=0pt,noitemsep]
    \item 각 팀은 1 이상 100 이하의 서로 다른 정수를 3개 선택해서 제출한다.
    \item 두 팀 이상이 선택한 수를 모두 제거한다. 이때 제출된 수들이 모두 제거되면 특별상은 사회에 환원된다.
    \item 100에 가장 가까운 수를 선택한 팀이 특별상의 주인공이 된다.
\end{enumerate}

예를 들어, A팀이 $(1, 50, 100)$, B팀이 $(2, 99, 100)$, C팀이 $(3, 97, 98)$, D팀이 $(2, 4, 50)$을 선택했다고 하자. 두 팀 이상이 선택한 $2, 50, 100$은 추첨에서 제외되므로, 남은 수 중 100과 가장 가까운 수인 99를 선택한 B팀이 특별상의 주인공이 된다.

당신은 은밀한 방법으로 다른 $N$개 팀이 선택한 수들을 전부 파악했다! 이제 남은 일은 제출할 3개의 정수를 최선을 다해 고르는 것뿐이다. 특별상을 노리기 위해, 가능한 조합들을 치밀하게 계산해보려 한다. 당신의 팀이 선택할 수 있는 정수 조합 중 특별상을 받을 수 있는 경우의 수를 구해보자. 단, 순서만 다른 조합은 같은 경우로 센다.

\InputFile
첫째 줄에 자신의 팀을 제외한 참가 팀의 수 $N$이 주어진다.

둘째 줄부터 $N$개의 줄에 걸쳐 다른 팀에서 고른 3개의 수 $A_i, B_i, C_i$가 공백으로 구분되어 주어진다.

\OutputFile
당신의 팀이 선택할 수 있는 정수 조합 중 특별상을 받을 수 있는 경우의 수를 구해보자.

\Constraints

\begin{itemize}[topsep=0pt,noitemsep]
    \item $1 \leq N \leq 100$
    \item $1 \le A_i, B_i, C_i \le 100$ ($1 \le i \le N$)
    \item $A_i \ne B_i; B_i \ne C_i; C_i \ne A_i$ ($1 \le i \le N$)
    \item 입력으로 주어지는 수는 모두 정수이다.
\end{itemize}

\Examples

\begin{example}
\exmpfile{./example/01.in.txt}{./example/01.out.txt}%
\exmpfile{./example/02.in.txt}{./example/02.out.txt}%
\exmpfile{./example/03.in.txt}{./example/03.out.txt}%
\end{example}

\Note
첫 번째 예시에서 당신의 팀이 특별상을 받기 위해서는 99와 100을 선택해 제거한 뒤, 2 이상 98 이하의 정수를 추가로 선택하면 된다.

두 번째 예시에서 당신의 팀이 특별상을 받을 수 있는 경우의 수는 없지만, 특별상의 행방은 당신에게 달려있다.
\end{problem}