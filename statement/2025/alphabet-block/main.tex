\def\probtitle{알파벳 블록}
\def\probno{A} % 문제 번호

\begin{problem}{\probno{}. \probtitle{}}

S, C, O, N 4종류의 알파벳 블록을 가지고 놀던 도현이는 두 가지 사실을 알아냈다.

\begin{itemize}[topsep=0pt,noitemsep]
    \item 알파벳 O 블록 1개를 반으로 자르면 알파벳 C 블록 2개로 사용할 수 있다. 알파벳 C 블록 2개를 합쳐서 알파벳 O 블록 1개로 사용하는 것도 가능하다.
    \item 알파벳 S 블록 1개를 뒤집으면 알파벳 N 블록 1개로 사용할 수 있다. 알파벳 N 블록 1개를 뒤집어서 알파벳 S 블록 1개로 사용하는 것도 가능하다.
\end{itemize}

도현이는 교내 대회 홍보를 위해 문자열 ``SCON"과 ``SCCC"가 하나씩 들어있는 상자를 사람들에게 나눠주기로 했다. 도현이는 가지고 있던 블록들로 문자열을 만들어 최대한 많은 사람들에게 상자를 나눠주려 한다. 각 알파벳 블록의 개수가 주어졌을 때, 도현이가 나눠줄 수 있는 상자 개수의 최댓값을 구해보자.

\InputFile

첫째 줄에 S, C, O, N 알파벳 블록의 개수를 의미하는 $b_S$, $b_C$, $b_O$, $b_N$이 공백으로 구분되어 주어진다.

\OutputFile

도현이가 나눠줄 수 있는 상자 개수의 최댓값을 출력한다.

\Constraints

\begin{itemize}[topsep=0pt,noitemsep]
    \item $0 \le b_S, b_C, b_O, b_N \le 1\,000\,000$
    \item 입력으로 주어지는 수는 모두 정수이다.
\end{itemize}

\Example

\begin{example}
    \exmpfile{./example/01.in.txt}{./example/01.out.txt}%
    \exmpfile{./example/02.in.txt}{./example/02.out.txt}%
    \exmpfile{./example/03.in.txt}{./example/03.out.txt}%
\end{example}

\end{problem}