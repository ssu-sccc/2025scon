\def\probtitle{서로소 조합}
\def\probno{H} % 문제 번호

\begin{problem}{\probno{}. \probtitle{}}

서로 다른 $n$개의 원소 중 순서를 구분하지 않고 $r$개를 선택하는 경우의 수 $C(n, r)$은 $\displaystyle{C(n, r) = \frac{n!}{r!(n-r)!}}$으로 구할 수 있다.

$C(n_1, r_1)$과 $C(n_2, r_2)$이 주어졌을 때, 두 값이 서로소인지 구하는 프로그램을 작성해 보자.

\InputFile
첫째 줄에 테스트 케이스의 수 $T$가 주어진다.

이후 $T$줄에 걸쳐 네 개의 정수 $n_1, r_1, n_2, r_2$가 공백으로 구분되어 주어진다.

\OutputFile
각 테스트 케이스마다 한 줄에 하나씩 $C(n_1, r_1)$과 $C(n_2, r_2)$가 서로소라면 1, 아니면 0을 출력한다.

\Constraints

\begin{itemize}[topsep=0pt,noitemsep]
    \item $1 \le T \le 5\,000$
    \item $0 < n_1, n_2 \leq 5\,000$
    \item $0 \leq r_1 \leq n_1$
    \item $0 \leq r_2 \leq n_2$
    \item 입력으로 주어지는 수는 모두 정수이다.
\end{itemize}

\Examples

\begin{example}
\exmpfile{./example/01.in.txt}{./example/01.out.txt}%
\end{example}

\end{problem}