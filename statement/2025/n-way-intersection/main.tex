\def\probtitle{$N$거리 건너기}
\def\probno{D} % 문제 번호

\begin{problem}{\probno{}. \probtitle{}}

\textit{한결이는 오늘도 학교에 간다. 왜냐하면 한결이는 졸업과 동시에 다시 입학하기 때문이다.}

\textit{한결이의 친구 창민이는 횡단보도가 있을 때 하나의 횡단보도를 삼거리, 사거리, 더 나아가 ``$N$거리"로 만들 수 있는 마법을 할 수 있다. 한결이를 놀리고 싶은 창민이는 한결이가 등교할 때 항상 지나가는 횡단보도를 골라 ``$N$거리"로 바꿔버렸다.}

``$N$거리"는 정$N$각형 모양의 교차로로, 시계 방향으로 각각 $1, 2, \cdots, N$번까지의 번호가 붙은 $N$개의 인도와, 아래와 같이 두 인도를 연결하는 $N$개의 횡단보도로 구성되어 있다.

\begin{itemize}[topsep=0pt,noitemsep]
    \item $i (1 \le i \le N-1)$번 횡단보도는 $i$번 인도와 $i+1$번 인도를 연결한다.
    \item $N$번 횡단보도는 $N$번 인도와 $1$번 인도를 연결한다.
\end{itemize}

횡단보도에는 신호등이 하나씩 있다. 신호등은 아래 규칙에 따라 돌아가면서 $1$초씩 초록불이 켜진다.

\begin{itemize}[topsep=0pt,noitemsep]
    \item 처음에는 $A_1$번 횡단보도의 신호등의 초록불이 켜진다.
    \item $A_i$번 횡단보도의 신호등의 초록불이 꺼짐과 동시에 $A_{i+1}$번 횡단보도의 신호등의 초록불이 켜진다.
    \item 단, $i = N$ 일 때는 $A_N$번 횡단보도의 신호등의 초록불이 꺼짐과 동시에 $A_1$번 횡단보도의 신호등의 초록불이 켜진다.
\end{itemize}

한결이는 등교할 때마다 $1$번 인도에서 출발하여 횡단보도를 건너 $M$번 인도로 가야 한다. 한결이는 걸음이 매우 빨라서 인도와 횡단보도를 걷는 시간을 무시할 수 있다고 할 때, 시계 방향과 반시계 방향 중 어느 방향으로 이동해야 $M$번 인도에 더 빨리 도착할 수 있는지 구해보자.

\InputFile

첫째 줄에 횡단보도의 개수를 의미하는 $N$, 한결이가 가야 하는 인도의 번호 $M$이 공백으로 구분되어 주어진다.

둘째 줄에 $N$거리 신호등의 초록불이 켜지는 순서를 의미하는 $N$개의 정수 $A_1, A_2, \cdots, A_N$이 공백으로 구분되어 주어진다.

\OutputFile

한결이가 $1$번 인도에서 $M$번 인도로 가기 위해 반시계 방향을 선택하는 것이 더 빠르다면 `\t{CCW}', 시계 방향을 선택하는 것이 더 빠르다면 `\t{CW}', 두 방향의 소요 시간이 같다면 `\t{EQ}'를 따옴표를 제외하고 출력한다.

\Constraints

\begin{itemize}[topsep=0pt,noitemsep]
    \item $3 \le N \le 100\,000$
    \item $2 \le M \le N$
    \item $1 \le A_i \le N (1 \le i \le N)$
    \item 수열 $A$의 원소는 서로 다르다. 즉, $i \ne j$ 이면 $A_i \ne A_j$이다.
    \item 입력으로 주어지는 수는 모두 정수이다.
\end{itemize}

\Example

\begin{example}
    \exmpfile{./example/01.in.txt}{./example/01.out.txt}%
    \exmpfile{./example/02.in.txt}{./example/02.out.txt}%
\end{example}

\Notes
총 소요 시간이 32비트 정수 범위를 넘을 수 있으므로, C/C++에서는 \t{long long}, Java에서는 \t{long}과 같은 자료형을 사용하는 것을 권장한다.

\begin{tikzpicture}

% Nodes
\node[circle, draw, minimum size=1cm] (A) at (0, 5) {1};
\node[circle, draw, minimum size=1cm] (B) at (5, 5) {2};
\node[circle, draw, minimum size=1cm, thick] (C) at (5, 0) {3};
\node[circle, draw, minimum size=1cm] (D) at (0, 0) {4};

% Edges
\draw (A) -- (B);
\draw (B) -- (C);
\draw (C) -- (D);
\draw (D) -- (A);

% Texts on edges
\node[above] at (2.5, 5.2) {\textbf{\underline{1}}, 5, 9, ...}; % 위쪽 간선
\node[right] at (5.2, 2.5) {0, \textbf{\underline{4}}, 8, ...}; % 오른쪽 간선
\node[below] at (2.5, -0.2) {\textbf{\underline{3}}, 7, 11, ...}; % 아래쪽 간선
\node[left] at (-0.2, 2.5) {\textbf{\underline{2}}, 6, 10, ...}; % 왼쪽 간선

\end{tikzpicture}

두 번째 예시에서 시계 방향으로 이동하면 1번 횡단보도와 2번 횡단보도를 각각 1초, 4초 뒤에 건널 수 있고, 반시계 방향으로 이동하면 4번 횡단보도와 3번 횡단보도를 각각 2초, 3초 뒤에 건널 수 있다. 따라서 반시계 방향으로 이동하는 것이 3번 인도에 더 빠르게 도착할 수 있다.

\end{problem}