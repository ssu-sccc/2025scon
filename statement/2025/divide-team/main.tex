\def\probtitle{대결}
\def\probno{I} % 문제 번호

\begin{problem}{\probno{}. \probtitle{}}

$N$명의 선수가 0번 팀과 1번 팀으로 나누어 대결을 한다. $i(1 \le i \le N)$번 선수는 $A_i$번 팀에 속해있으며, 능력치는 $B_i$이다. (단, $A_i$는 0 또는 1)

대결의 진행자인 당신은 대결을 여러 라운드에 걸쳐 진행하려고 한다. 당신은 $1$개 이상 $N$개 이하의 라운드를 자유롭게 만든 뒤, 아래 규칙에 따라 라운드에 참가할 선수를 선택해야 한다.

\begin{itemize}[topsep=0pt,noitemsep]
    \item 각 라운드에는 연속된 번호의 선수들만 참가할 수 있다.
    \item 모든 선수는 정확히 하나의 라운드에 참가해야 한다.
    \item 모든 라운드는 한 명 이상의 선수가 참가해야 한다.
\end{itemize}

예를 들어, $(1, 2), (3), (4, 5)$와 같이 라운드를 구성할 수 있지만, $(1, 3), (2, 4, 5)$와 같이 불연속적인 번호의 선수들로 라운드를 구성하거나, $(1, 2, 3), (3, 4)$와 같이 한 선수가 0개 또는 2개 이상의 라운드에 참가하도록 라운드를 구성할 수 없다.

대결에서 각 팀의 점수는 다음과 같이 계산한다.

\begin{itemize}[topsep=0pt,noitemsep]
    \item 참가하는 선수의 번호가 작은 라운드부터 차례대로 $1, 2, \cdots$번째 라운드라고 하자.
    \item $j$번째 \textbf{라운드의 점수}는 해당 라운드에 참가한 팀 소속 선수들의 능력치의 합에 $j$를 곱한 값이다.
    \item 팀의 \textbf{최종 점수}는 해당 팀이 모든 라운드에서 얻은 점수의 총합이다.
\end{itemize}

예를 들어, $A = [1, 1, 0, 0, 1]$, $B = [1, 2, 3, 4, 5]$이고, 라운드를 $(1, 2), (3), (4, 5)$로 구성했다고 하자. 0번 팀은 2 라운드에서 $3 \times 2 = 6$점, 3 라운드에서 $4 \times 3 = 12$점을 획득해서 최종 점수는 18점이고, 1번 팀은 1 라운드에서 $(1+2) \times 1 = 3$, 3 라운드에서 $5 \times 3 = 15$점을 획득해 최종 점수는 18점이다.

1번 팀의 열렬한 팬인 당신은 1번 팀의 최종 점수에서 0번 팀의 최종 점수를 뺀 값 $S$를 최대화하려고 한다. 만약 $S$를 최대화하는 라운드 구성 방법이 여러가지라면, 라운드 개수 $M$이 최소가 되도록 라운드를 구성해야 한다.

만들 수 있는 최대 점수 차이 $S$와, 이를 달성하기 위한 가장 적은 라운드의 개수 $M$을 구하는 프로그램을 작성해 보자.

\InputFile
첫째 줄에 선수의 수를 의미하는 정수 $N$이 주어진다.

둘째 줄에 각 선수의 소속 팀을 나타내는 $N$개의 정수 $A_1, A_2, \cdots, A_N$ 이 공백으로 구분되어 주어진다.

셋째 줄에 각 선수의 능력치를 나타내는 $N$개의 정수 $B_1, B_2, \cdots, B_N$ 이 공백으로 구분되어 주어진다.

\OutputFile
1번 팀의 최종 점수에서 0번 팀의 최종 점수를 뺀 값이 최대가 되도록 하는 최소 라운드 개수 $M$과 그때의 최대 점수 차 $S$를 공백으로 구분해서 출력한다.

\newpage
\Constraints
\begin{itemize}[topsep=0pt,noitemsep]
    \item $1 \le N \le 1\,000\,000$
    \item $0 \le A_i \le 1$ ($1 \le i \le N$)
    \item $0 \le B_i \le 1\,000\,000$ ($1 \le i \le N$)
    \item 입력으로 주어지는 수는 모두 정수이다.
\end{itemize}

\Examples

\begin{example}
\exmpfile{./example/01.in.txt}{./example/01.out.txt}%
\exmpfile{./example/02.in.txt}{./example/02.out.txt}%
\end{example}

\Notes

정답이 32비트 정수 범위를 넘을 수 있으므로, C/C++에서는 \t{long long}, Java에서는 \t{long}과 같은 자료형을 사용하는 것을 권장한다.

\end{problem}