{
    \indent
    \Large
    
    언어 가이드
}

\begin{itemize}[noitemsep]

    \item 채점은 Intel Xeon E5-2666v3 프로세서를 사용하는 AWS EC2 c4.large 인스턴스에서 진행합니다.
    \item 채점 서버의 운영체제는 Ubuntu 16.04.7 LTS 입니다.
    \item 아래 언어 중 원하는 언어를 선택해 사용할 수 있습니다.
    
    \begin{itemize}[noitemsep,topsep=0pt]
        \item C11: \texttt{gcc 11.1.0} % \\
        % 컴파일: \texttt{gcc Main.c -o Main -O2 -Wall -lm -static -std=gnu11}\\\texttt{-DONLINE\_JUDGE -DBOJ}\\
        % 실행: \texttt{./Main}
        \item C++17: \texttt{g++ 11.1.0} % \\
        % 컴파일: \texttt{g++ Main.cc -o Main -O2 -Wall -lm -static -std=gnu++17}\\\texttt{-DONLINE\_JUDGE -DBOJ}\\
        % 실행: \texttt{./Main}
        \item Java 15: \texttt{OpenJDK version "16.0.1" 2021-04-20} % \\
        % 컴파일: \texttt{javac -release 15 -J-Xms1024m -J-Xmx1920m -J-Xss512m}\\\texttt{-encoding UTF-8 Main.java}\\
        % 실행: \texttt{java -Xms1024m -Xmx1920m -Xss512m -Dfile.encoding=UTF-8 -XX:+UseSerialGC}\\\texttt{-DONLINE\_JUDGE=1 -DBOJ=1 Main}
        \item Python 3: \texttt{Python 3.13.1} % \\
        % 컴파일: \texttt{python3 -W ignore -c "import py\_compile; py\_compile.compile(r'Main.py')"}\\
        % 실행: \texttt{python3 -W ignore Main.py}
        \item PyPy3: \texttt{Python 3.10.14, PyPy 7.3.12 with GCC 10.2.1 20210130} %\texttt{(Red Hat 10.2.1-11)} % \\
        % 컴파일: \texttt{pypy3 -W ignore -c "import py\_compile; py\_compile.compile(r'Main.py')"}\\
        % 실행: \texttt{pypy3 -W ignore Main.py}
        \item 컴파일과 실행 옵션은 \texttt{https://help.acmicpc.net/language/info}에서 확인할 수 있습니다.
        
    \end{itemize}

    \item C11/C++17에서 \texttt{scanf\_s}와 \texttt{Windows.h}등의 비표준 함수를 사용할 수 없습니다.
    \item Java를 사용하는 경우, \texttt{main} 메소드를 포함하는 클래스의 이름은 \texttt{Main}이어야 합니다.
    \item Python에서 \texttt{numpy}와 같은 외부 모듈을 사용할 수 없습니다.
    \item 채점 사이트에서 컴파일 에러를 받은 경우, `컴파일 에러' 글씨를 누르면 오류가 발생한 위치를 볼 수 있습니다.

    \item 아래 코드는 표준 입력(standard input)을 통해 공백으로 구분된 두 정수를 입력으로 받아서 표준 출력(standard output)을 통해 합을 출력하는 코드입니다.

% 어떻게든 에러를 좀 조져 보려는 키파의 노력
% \renewcommand{\inputminted}{{\Huge\textbf{\textcolor{red}{TODO: there once was a code here\ldots}}}}

    \begin{itemize}[noitemsep]
        \item C11%
        \inputminted[frame=lines,baselinestretch=1.2,linenos]{c}{rule-example-code/c11-1000.c}
        \item C++17%
        \inputminted[frame=lines,baselinestretch=1.2,linenos]{cpp}{rule-example-code/cpp17-1000.cpp}
        \item Java 15%
        \inputminted[frame=lines,baselinestretch=1.2,linenos]{java}{rule-example-code/java-1000.java}
        \item Python 3 / PyPy3%
        \inputminted[frame=lines,baselinestretch=1.2,linenos]{python}{rule-example-code/py3-1000.py}
    \end{itemize}

    \item 입출력 양이 많을 때는 위 코드를 사용한 입출력이 너무 오래 걸리기 때문에 다른 방식으로 입출력해야 합니다.

    \item C11/C++17에서 \texttt{scanf}와 \texttt{printf}를 사용하는 경우, 입출력 속도는 문제를 해결할 수 있을 정도로 충분히 빠릅니다.
    \item C++17에서 \texttt{cin}과 \texttt{cout}을 사용하는 경우, 입출력 전에 \texttt{ios\_base::sync\_with\_stdio(false);}와 \texttt{cin.tie(nullptr);}를 사용하여야 합니다. 단, 이 이후에는 \texttt{cin}, \texttt{cout} 계열 함수와 \texttt{scanf}, \texttt{printf} 계열 함수를 섞어서 사용하면 안 됩니다. 또한, 개행문자로 \texttt{std::endl} 대신 \texttt{"\char92{}n"}을 사용해 주세요. 

    \item Java 15에서는 \texttt{BufferedReader}와 \texttt{BufferedWriter}를 사용하여야 합니다.
    
    \item Python 3 및 PyPy3 에서는 \texttt{input()} 대신 \texttt{sys.stdin.readline().rstrip("\char92{}n")}을 사용하여야 합니다. 코드의 가장 위 부분에 \texttt{import sys} 와\\
    \texttt{input = lambda: sys.stdin.readline().rstrip("\char92{}n")} 을 사용하여야 합니다.
    
    \item 아래 코드는 표준 입력(standard input)을 통해 문제의 개수 $T$를 입력받은 다음 $T$줄에 걸쳐 공백으로 구분된 두 정수를 입력으로 받아 표준 출력(standard output)을 통해 두 정수의 합을 총 $T$줄에 걸쳐 출력하는 코드입니다.

    \begin{itemize}[noitemsep]
        \item C++17%
        \inputminted[frame=lines,baselinestretch=1.2,linenos]{cpp}{rule-example-code/cpp17-15552.cpp}
        \item Java 15%
        \inputminted[frame=lines,baselinestretch=1.2,linenos]{java}{rule-example-code/java-15552.java}
        \item Python 3 / PyPy3%
        \inputminted[frame=lines,baselinestretch=1.2,linenos]{python}{rule-example-code/py3-15552.py}
    \end{itemize}
	
\end{itemize}